\documentclass[conference]{IEEEtran}
\IEEEoverridecommandlockouts
% The preceding line is only needed to identify funding in the first footnote. If that is unneeded, please comment it out.
\usepackage{cite}
\usepackage{amsmath,amssymb,amsfonts}
\usepackage{algorithmic}
\usepackage{graphicx}
\usepackage{textcomp}
\usepackage{xcolor}
\def\BibTeX{{\rm B\kern-.05em{\sc i\kern-.025em b}\kern-.08em
    T\kern-.1667em\lower.7ex\hbox{E}\kern-.125emX}}


\usepackage{subcaption}

\usepackage{tikz}
\usetikzlibrary{calc}
\usetikzlibrary{arrows.meta}

\usepackage{math}
\usepackage{temporal}

\begin{document}

\title{Adaptive Multi-Agent Programming with Programs and Promises
(Doctoral Symposium)}

\author{\IEEEauthorblockN{Oleks Shturmov}
\IEEEauthorblockA{\textit{Department of Informatics} \\
\textit{University of Oslo}\\
Norway \\
\texttt{oleks@oleks.info}}
% \and
% \IEEEauthorblockN{2\textsuperscript{nd} Given Name Surname}
% \IEEEauthorblockA{\textit{dept. name of organization (of Aff.)} \\
% \textit{name of organization (of Aff.)}\\
% City, Country \\
% email address or ORCID}
% \and
% \IEEEauthorblockN{3\textsuperscript{rd} Given Name Surname}
% \IEEEauthorblockA{\textit{dept. name of organization (of Aff.)} \\
% \textit{name of organization (of Aff.)}\\
% City, Country \\
% email address or ORCID}
% \and
% \IEEEauthorblockN{4\textsuperscript{th} Given Name Surname}
% \IEEEauthorblockA{\textit{dept. name of organization (of Aff.)} \\
% \textit{name of organization (of Aff.)}\\
% City, Country \\
% email address or ORCID}
% \and
% \IEEEauthorblockN{5\textsuperscript{th} Given Name Surname}
% \IEEEauthorblockA{\textit{dept. name of organization (of Aff.)} \\
% \textit{name of organization (of Aff.)}\\
% City, Country \\
% email address or ORCID}
% \and
% \IEEEauthorblockN{6\textsuperscript{th} Given Name Surname}
% \IEEEauthorblockA{\textit{dept. name of organization (of Aff.)} \\
% \textit{name of organization (of Aff.)}\\
% City, Country \\
% email address or ORCID}
}

\maketitle

\begin{abstract}

% Programming adaptive systems, composed of physically distributed
% agents, entails programming the adaptation logic of the individual
% agents.

Agents that adapt their behavior in response to communicative events
by other agents, can be viewed as machines that can be programmed by
those other agents. At the time same time, such agents can be viewed
as machines that each deliver a promise to behave a certain way, until
given conditions change. This work explores the potential benefits for
the programmer in taking this perspective, when programming multiple,
adaptive, physically distributed agents, engaged in a common
objective.

The approach is to have agents communicate sentences in linear
temporal logic (LTL) as to their projected behaviour going forth,
depending on subsequent events. Events may be induced by changes in
their environment, or messages from other agents. Armed with LTL
sentences about the projected behaviour of other agents, and agent can
employ a live synthesis technique in attempt to reach its objective.

\end{abstract}

\begin{IEEEkeywords}
adaptive, multi-agent, distributed, programming, linear temporal logic
\end{IEEEkeywords}

\section{Motivation and Challenges}

\emph{TASK: explain why this research is important, and identify the
key research question(s) and challenges that have yet to be addressed
by the community - including limitations of current approaches.}

% Contemporary distributed systems communication primitives tend to
% deal in short-term planning horizons. For instance, while an inquiry
% about the state of an agent may resolve to a particular response,
% its state may be subject to fast-paced change, when compared with
% the time it takes to communicate across the network. This renders
% the response relevant only for a short period of time, if at all,
% when delivered.  Follow-up requests will be relevant for an even
% shorter period of time.

% Uncontested this leads to either high failure rates of follow-up
% requests, or highly congested networks. In particular, the state of
% a remote agent may be ill-predictable, while short-sighted status
% messages have little positive impact on the expected success rate of
% follow-up messages. Hence, we may choose to flood the network in
% attempt to keep track of the fast-paced changes, while still gaining
% little confidence that we can act on those changes in due time.

%% At the same time, agents in highly dynamic systems often cannot freeze
%% in anticipation of follow-up requests.

% As a particular example

Consider the following multi-agent problem: 3 robots roam a fixed-size
room, looking for tasks to do. A task can be solved by 2 robots
getting within radius $\omega$ of the task. Tasks appear at random, in
random positions, and neither move, nor disappear until solved. Agents
have an omnidirectional field of sight with radius $\omega$, and can
communicate by broadcasting messages within radius $\beta$, where
$\beta>\omega$. The robots roam around, looking for tasks to do. When
a robot spots a task, it stops and calls out for help.

% calling out for help from other agents whenever they see a task.

\tikzset{
  unsuccessful/.style={
    draw=gray,dashed
  }
}

\begin{figure}
\centering
\begin{subfigure}{0.4\textwidth}
\centering
\begin{tikzpicture}
  \coordinate (a) at (0,0);
  \coordinate (b) at (0,2);
  \coordinate (c) at (2,0);
  \coordinate (d) at (2,2);
  \coordinate (e) at (4,0);
  \coordinate (f) at (4,2);
  \draw %
    (a) -- (b)node[pos=1.1,scale=0.8]{Alice}%
    (c) -- (d)node[pos=1.1,scale=0.8]{Bob}%
    (e) -- (f)node[pos=1.1,scale=0.8]{Carol};
  \draw[stealth-,shorten <=8mm,unsuccessful] %
    ($(a)!0.8!(b)$) -- node[above,near end,scale=0.6]{\texttt{help}}($(c)!0.8!(d)$);
  \draw[stealth-,shorten <=5mm,unsuccessful] %
    ($(a)!0.6!(b)$) -- node[above,near end,scale=0.6]{\texttt{help}}($(c)!0.6!(d)$);
  \draw[stealth-] %
    ($(a)!0.4!(b)$) -- node[above,near end,scale=0.6]{\texttt{help}}($(c)!0.4!(d)$);

  \draw[-stealth] %
    ($(a)!0.2!(b)$) -- node[above,near end,scale=0.6]{\texttt{omw}}($(c)!0.2!(d)$);

  \draw[stealth-,shorten <=2mm,unsuccessful] %
    ($(e)!0.8!(f)$) -- node[above,near end,scale=0.6]{\texttt{help}}($(c)!0.8!(d)$);
  \draw[stealth-,shorten <=6mm,unsuccessful] %
    ($(e)!0.6!(f)$) -- node[above,near end,scale=0.6]{\texttt{help}}($(c)!0.6!(d)$);
  \draw[stealth-,shorten <=8mm,unsuccessful] %
    ($(e)!0.4!(f)$) -- node[above,near end,scale=0.6]{\texttt{help}}($(c)!0.4!(d)$);

\end{tikzpicture}
\caption{Alice moves closer to Bob, while Carol moves away.}
\label{fig1_1}
\end{subfigure}

\begin{subfigure}{0.4\textwidth}
\centering
\begin{tikzpicture}
  \coordinate (a) at (0,0);
  \coordinate (b) at (0,2);
  \coordinate (c) at (2,0);
  \coordinate (d) at (2,2);
  \coordinate (e) at (4,0);
  \coordinate (f) at (4,2);
  \draw %
    (a) -- (b)node[pos=1.1,scale=0.8]{Alice}%
    (c) -- (d)node[pos=1.1,scale=0.8]{Bob}%
    (e) -- (f)node[pos=1.1,scale=0.8]{Carol};
  \draw[stealth-,shorten <=8mm,unsuccessful] %
    ($(a)!0.8!(b)$) -- node[above,near end,scale=0.6]{\texttt{help}}($(c)!0.8!(d)$);
  \draw[stealth-,shorten <=4mm,unsuccessful] %
    ($(a)!0.6!(b)$) -- node[above,near end,scale=0.6]{\texttt{help}}($(c)!0.6!(d)$);
  \draw[stealth-] %
    ($(a)!0.4!(b)$) -- node[above,near end,scale=0.6]{\texttt{help}}($(c)!0.4!(d)$);

  \draw[-stealth] %
    ($(a)!0.2!(b)$) -- node[above,near end,scale=0.6]{\texttt{omw}}($(c)!0.2!(d)$);

  \draw[stealth-,shorten <=4mm,unsuccessful] %
    ($(e)!0.8!(f)$) -- node[above,near end,scale=0.6]{\texttt{help}}($(c)!0.8!(d)$);
  \draw[stealth-,shorten <=2mm,unsuccessful] %
    ($(e)!0.6!(f)$) -- node[above,near end,scale=0.6]{\texttt{help}}($(c)!0.6!(d)$);
  \draw[stealth-] %
    ($(e)!0.4!(f)$) -- node[above,near end,scale=0.6]{\texttt{help}}($(c)!0.4!(d)$);

  \draw[-stealth] %
    ($(e)!0.2!(f)$) -- node[above,near end,scale=0.6]{\texttt{omw}}($(c)!0.2!(d)$);

\end{tikzpicture}
\caption{Alice and Carol move towards Bob.}
\label{fig1_2}
\end{subfigure}
\caption{Sequence diagrams of Bob's calls out for
\label{fig1}
\texttt{help}}
\end{figure}

Figure \ref{fig1_1} illustrates the communication between 3 robots,
Alice, Bob, and Carol, where Bob has found a task and calls out to
Alice and Carol for \texttt{help}. As Alice and Carol roam about the
room, they may be out of reach for Bob's broadcasting signal. Hence,
Bob has to repeat the signal until eventually getting a response.

Figure \ref{fig1_2} presents a slightly less happy path, where both
Alice and Carol get within reach of Bob. In this scenario, Bob ought
to abort the \texttt{help} request from either Alice or Carol.
Choosing which request to cancel is best informed, if we have an
indication of how quickly each would arrive.

To expand the relevance horizon of our messages, consider: (1)
constructing responses such that they give an indication of not just
the current state of an agent, but also its possible future states,
and/or (2) composing follow-up requests such that they can be serviced
successfully, despite possible meanwhile state changes at the remote
agent.

To this end, both requests and responses may employ sentences in a
temporal logic, a logic that deals with truth of propositions over
time\cite{1977-Temporal-Logic}. Temporal logics have often been used
as specification languages for concurrent systems, used for model
checking and
synthesis\cite{2021-Specification-Patterns-for-Robotic-Missions}.

However, previous work has mainly focused on global, or
decentralized\cite{2016-Multi-Agent-Planning-under-Local-LTL-Specs-and-Event-Based-Sync,
2018-Multi-robot-LTL-Planning-Under-Uncertainty}, but fixed temporal
logic specifications. Although some work exists on supporting changing
specifications at runtime\cite{2022-Live-Synthesis}, existing work
falls short of considering temporal logic sentences as the language of
message exchange between agents, and full-blown runtime synthesis on
the basis of the exchange of such messages.

Possibly relevant: existing work shows that techniques from runtime
verification can simplify otherwise complex static synthesis
tasks\cite{2022-Runtime-Verification-Meets-Controller-Synthesis},


\section{Contribution and Objectives}

\emph{TASK: present the main contribution of the research in a way
that a non-expert could clearly understand, then present the detailed
objectives of your research, highlighting why your work is novel in
comparison to existing research.}


%Armed with such sentences for the surrounding agents, a given agent
%can synthesize a plan to proceed towards its objective at
%runtime\cite{2022-Live-Synthesis}.

A \texttt{help} message from Bob ($B$) looks as follows:

\begin{align}
\mathtt{help}_B \until %
\p{\bigvee_{i=1}^n %
  \mathtt{omw}_i^B \until %
  \p{\mathtt{nearby}_i^B \wedge \mathtt{work}_i \wedge \mathtt{work}_B} %
} %
\end{align}

Meaning, Bob will be sending out \texttt{help} messages until one of
the other agents responds back with a message \texttt{omw}, coupled
with a promise that they will eventually get near to Bob, and do their
part of the work. Bob also promises to do his part of the work, in the
end.

An initial \texttt{omw} response from Alice ($A$) may then look as
follows:

\begin{align}
\mathtt{omw}^B_A \until %
  \p{\mathtt{nearby}_A^B \wedge \mathtt{work}_A}%
\end{align}

Meaning, Alice will be sending out \texttt{omw} messages until it
eventually reaching Bob and doing her part of the work.

% 2
% vacuuming and 1 polishing robot roam a fixed-size room looking for
% tasks to do---spots to vacuum and polish. A task spawns at a random
% point in time, at a random position, and neither moves, nor disappears
% until a vacuuming and polishing robot get within radius $\omega$ of
% the task and solve it.

% It seems safe to conjecture that an approach to communication that
% seeks to increase the expected success rate of follow-up requests
% would lead to less congested network traffic.

\section{Methodology and Preliminary Results}

\emph{TASK: present the methodology of your approach, discussing why
it is suitable to your contribution, and also present any preliminary
results if you have them, including a statement about the current
status of your work.}

\section{Future and Research Plan}

\emph{TASK: summarise your intended future work and provide a schedule
of milestones with a discussion of their feasibility.}

Implement a MAS programming environment, where (1) agents communicate
with LTL sentences, embedded in both their requests and responses, (2)
have their own goals specified as LTL sentences, and (3) each run an
LTL-based synthesis procedure to iteratively arrive at the next step
towards fulfilling their goal.

\bibliographystyle{IEEEtran}
\bibliography{references}

\end{document}
